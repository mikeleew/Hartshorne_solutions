    \subsection*{1.1 (a)   {edealba}}
    
    \hspace{1cm} Note that $A(Y)$ is the coordinate ring for the curve $y-x^2 = 0$, and since we're working over the plane, suppose we're working on $\A^2$ (affine plane) over the field $k$, such that the coordinate ring of the plane is just $k[x, y]$. We want to show that $A(Y) \cong k[x]$.

Note that the coordinate ring $A(Y)$ is defined as the following quotient ring:
$$ A(Y) = k[x, y]/J, $$
where $J = (y-x^2)$, i.e. $J$ is the ideal generated by $y-x^2$. Consider the following mapping:
$$ \phi: k[x, y] \to k[x], $$
where $\phi$ is an evaluation mapping on $y$, where $y \mapsto x^2$. Evaluation maps like $\phi$ are ring homomorphisms. We will show that $\ker(\phi)$ is exactly equivalent to $J$ since all polynomials generated by $y-x^2$ get mapped to $x^2-x^2 = 0$ via $\phi$, hence $J \subset \ker(\phi)$. To show equality, we need to show that $\ker(\phi) \subset J$. Let $a \in \ker(\phi)$, i.e. $\phi(a) = 0$. We want to show that $a \in J$, i.e. that we can write $a$ as:
$$a = A(x, y) \cdot (y-x^2), $$
where $A(x, y) \in k[x, y]$. If $a$ were \emph{not} in $J$ then we would have some non-zero remainder $r(x, y)$ with degree of $y$ less than $1$ (since the degree of $y$ in $y-x^2$ is $1$), so we can write the remainder as $r(x)$ such that:
$$a = A(x, y) \cdot (y-x^2) + r(x). $$
Let's apply $\phi$ to the RHS: $\phi(A(x, y) \cdot (y-x^2) + r(x)) = A(x, x^2)\cdot (x^2-x^2) + r(x) = A(x, x^2)\cdot 0 + r(x) = r(x)$. Note that this implies that $\phi(a) = r(x) = 0$. This means that the remainder is zero, and hence $a = A(x, y)\cdot(y-x^2)$ which means that $a \in J$ like we wanted to show. Thus, $\ker(\phi) = J$.

Next, let's show that $\phi$ is surjective. Let $p(x) \in k[x]$ be an arbitary polynomial of degree $m$, so we can write:
$$ p(x) = \sum_{i=0}^{m} c_ix^{i} $$
Note that $c_i \in k \subset k[x] \subset k[x, y]$ and also $x^i \in k[x] \subset k[x, y]$ for all $i$, then clearly $p(x) \in k[x, y]$ where $\phi(p(x)) = p(x)$. Since $\phi$ is surjective, then we know that $\im(\phi) = k[x]$. By the first isomorphism theorem, we get:
$$ k[x, y] / \ker(\phi) \cong \im(\phi), $$
where $\ker(\phi) = J$ and $\im(\phi) = k[x]$, thus
$$ k[x, y] / J \cong k[x],$$
and finally since we know $A(Y) = k[x, y] / J$, then clearly we get that $A(Y) \cong k[x]$, so we've shown that the coordinate ring $A(Y)$ is, indeed, isomorphic to the polynomial rings in one variable over $k$. \qed
