  \subsection*{1.4   {edealba}}

  Consider the diagonal line $y=x$ in $\A^2$ where we're working over the coordinate ring $k[x, y]$. Note that $V(y-x) \subset \A^2$ is closed in the Zariski topology of $\A^2$, and is the following:
$$ V(y-x) = \left\{ p \in k^2 : f(p) = 0 \text{ for all } f \in (y-x) \right\} $$
However, when we look at the closed sets in $\A^1$, we only have the following: (1) sets with a finite number of points, (2) the entire affine line $\A^1$. In $\A^1 \times \A^1$ we have: (1) also sets with finite number of points, (2) the entire space $\A^1 \times \A^1$, and (3) we also finite unions of vertical and horizontal lines from $\A^1 \times \{ p_i \}$ and $\{ p_j \} \times \A^1$ for points $p_i$ and $p_j$ in $\A^1$. Note that it's \emph{impossible} to construct the diagonal line $y-x=0$ using any combination of closed sets in $\A^1 \times \A^1$, we would need an infinite number of points which is \emph{not} allowed in the Zariski topology. Thus, although $\A^2$ and $\A^1 \times \A^1$ may be identical to each other, their topologies are not the same. \qed

