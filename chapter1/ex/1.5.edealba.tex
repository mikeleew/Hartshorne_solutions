  \subsection*{1.5   {edealba}}

  Let $B$ be a $k$-algebra, and let $W = V(J) \subset \A^n$ be some algebraic set of $\A^n$ for some ideal $J \subset k[x_1, \ldots, x_n]$, where $W$ is the set of all points that vanish for all polynomial functions in $J$. We then have the following coordinate ring (which is the following quotient ring):
$$ A(W) = k[x_1, \ldots, x_n] / I(W),$$
and since $W = V(J)$, by Hilbert's strong Nullstellensatz, we know that $I(V(J)) = \sqrt{J}$, hence
$$ A(W) = k[x_1, \ldots, x_n] / \sqrt{J} $$
We want to show that $B \cong A(W)$ iff $B$ is a finitely generated $k$-algebra with no nilpotent elements.

($\Rightarrow$) First, suppose $B \cong A(W)$. Since we've defined $A(W)$ to be the quotient ring of $k[x_1, \ldots, x_n]$ modded out by the radical of an ideal $\sqrt{J}$, then this is automatically a finitely generated $k$-algebra. We want to show that it has no nilpotent elements, i.e. there is \emph{no} non-zero element $f \in B \cong A(W)$ such that $f^m = 0$ for some $m \geq 2$.

Suppose for contradiction that there exists some nilpotent element $f \in B \cong A(W)$ such that $f^m = 0$ for some $m \geq 2$. Since $f$ is assumed to be non-zero, this means that $f \notin I(W) = \sqrt{J}$. Since $J \subset \sqrt{J}$, then $f \notin \sqrt{J}$ implies that $f \notin J$. Note that since $f \notin I(W)$ then this also means that there exists some point $p' \in W$ such that $f(p') \neq 0$. Next, since we have $f^m = 0$, then $f^m \in I(W) = \sqrt{J}$. This means that $f^m$ vanishes for all points $p \in W$, and hence $f^m(p') = 0$. Since $f^m(p') = 0$, this implies that $f^m(p') = (f(p'))^m = 0$ which implies that $f(p') =0$ which contradicts our previous $f(p') \neq 0$. With this, we've shown that there \emph{can't} be a nilpotent element $f \in B \cong A(W)$.

($\Leftarrow$) Let $B$ be a finitely generated $k$-algebra with generators $\{ t_1, \ldots, t_n \} \subset B$ such that any element in $B$ can be written as a polynomial in these generators, and $B$ also has no nilpotent elements. Consider the following mapping:
$$ \phi: k[x_1, \ldots, x_n] \to B, $$
where $\phi$ maps $x_i \mapsto t_i$ for all $1\leq i \leq n$. This mapping is clearly surjective since any polynomial $f \in B$ can be written by swapping each $t_i$ with its corresponding $x_i$. Note that if the generators $\{t_1, \ldots, t_n \}$ have some algebraic relations, this is captured by $\ker(\phi)$. Let $J = \ker(\phi) \subset k[x_1, \ldots, x_n]$ be the ideal generated by all the algebraic relations between the generators $\{ t_1, \ldots, t_n \}$. By the First isomorphism theorem, we get:
$$ k[x_1, \ldots, x_n]/J \cong B$$
Next, we want to show that $J$ is a radical ideal. Note that there are \emph{no} nilpotent elements in $B$. If there \emph{were} some nilpotent element then suppose $f \in B$ were nilpotent, then this would mean that $f^m = 0$ for some $m\geq 1$.

(1) An ideal $J$ is radical if for some $x^m \in J$ with $m \geq 1$ implies that $x \in J$, i.e.:
$$ x^m \in J \implies x \in J  $$

(2) Since $B \cong k[x_1, \ldots, x_n]/J$, then elements of $B$ are of the coset form $\bar{f} = f + J$. If $\bar{f} \in B$ were nilpotent then this means that $\bar{f}^m = 0$ for some $m\geq 1$, which means that $f^m + J = 0$ which implies that $f^m \in J$ for $f\in k[x_1, \ldots, x_n]$.

Note that since there are \emph{no} nilpotent elements in $B$, then this means for all non-zero $\bar{f} \in B$ we know that $\bar{f} = f + J \neq 0$, i.e. $f \notin J$ where $f^m \neq 0$ for all $m\geq 1$. Furthermore, note that the contrapositive of (1) means that if the following holds:
$$ x \notin J \implies x^m \notin J \text{ for all } m\geq 1, $$
then $J$ is a radical ideal. We know that for any non-zero $\bar{f}$ we have $f \notin J$, and since $\bar{f}$ is \emph{not} nilpotent, then $f^m \notin J$ for all $m\geq 1$. With this, we've shown that $B$ not having any nilpotent elements implies that $J$ is a radical ideal, i.e. $J = \sqrt{J}$.

Finally, let's consider the algebraic set $W = V(J) \subset \A^n$. From Hilbert's strong Nullstellensatz, we know that the coordinate ring $A(W)$ is the following:
$$ A(W) \cong k[x_1, \ldots, x_n]/I(V(J)) = k[x_1, \ldots, x_n]/\sqrt{J}, $$
but note that since $J$ is radical, then we get:
$$ A(W) \cong k[x_1, \ldots, x_n]/J, $$
so we get that $B \cong k[x_1, \ldots, x_n]/J \cong A(W)$ like we wanted to show. \qed

