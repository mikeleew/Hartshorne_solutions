\subsection*{2.1}
\addcontentsline{toc}{subsection}{2.1}

Prove the ``homogeneous Nullstellensatz,'' which says if $J \subseteq S$ is a homogeneous ideal, and if $f \in S$ is a homogeneous polynomial with $\deg f > 0$, such that $f(P) = 0$ for all $P \in Z(J)$ in $\P^n$, then $f^q \in J$ for some $q > 0$. [\textit{Hint:} Interpret the problem in terms of the affine $(n + 1)$-space whose affine coordinate ring is $S$, and use the usual Nullstellensatz, (1.3A).]
