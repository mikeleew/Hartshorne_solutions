\subsection*{2.10 (a) (b) {edealba}}

\begin{enumerate}
    \item We know that $Y \subseteq \P^n$ is a nonempty algebraic set, so there exists a homogeneous ideal $J$ such that $Z(J) = Y$. Note that since $Y = Z(J)$, then $C(Y) = C(Z(J))$. In \textbf{1.2.1} we showed that $C(Z(J)) = V(J) \subseteq \A^{n+1}$ for any ideal $J \subseteq k[x_0, \ldots, x_n]$, hence $C(Y) = V(J) \subseteq \A^{n+1}$. $C(Y) = V(J)$ is an algebraic set in $\A^{n+1}$ like we wanted to show.

Next, let's show that the ideal of $C(Y)$, i.e. $I(C(Y))$, is equal to $I(Y)$.

First, let's show that $I(Y) \subseteq I(C(Y))$. Let $f \in I(Y)$ such that $f$ is a \emph{homogeneous} polynomial of degree $d$, then this means that $f$ is a polynomial that vanishes at every point $p \in Y$. We want to show that $f \in I(C(Y))$. Let $q \in C(Y)$ be an arbitary point in the affine cone over $Y$. First case, suppose $q = (0, \ldots, 0)$. Since $I(Y)$ is the set of polynomials that vanish at every point in $Y \subseteq \P^n$, then these must be polynomials with \emph{zero} constant, hence $f(0, \ldots, 0) = 0$ (this is also clear since $f$ is homogeneous). Suppose $q = (a_0, \ldots, a_n)$ is an arbitary non-zero point in $C(Y)$ with some non-zero $a_i$. Note that, by definition of $I(Y)$, since $f$ vanishes at every \emph{projective} point in $Y$, then $f$ must vanish at \emph{all} of its representative points that exist in affine space. $q$ has a corresponding projective point $\overline{q} \in \P^n$ such that $f(\overline{q}) = 0$ which implies that $f(q) = 0$. If $\overline{q} = [q_0: \cdots : q_n] \in \P^n$, then $q$ is of the form:
$$ q = (\lambda q_0, \ldots, \lambda q_n), \text{ for } \lambda \in k, $$
and since $f$ is a homogeneous polynomial of degree $d$, then we can write:
$$ f(q) = f(\lambda q_0, \ldots, \lambda q_n) = \lambda^d f(q_0, \ldots, q_n) = \lambda^d \cdot 0 = 0, $$
so we see that any homogeneous polynomial $f \in I(Y)$ is in $I(C(Y))$. Furthermore, if we have some other \emph{non-homogeneous} polynomial $g \in I(Y)$, since $I(Y)$ is a homogeneous ideal, this means that it's generated by a set of homogeneous polynomials. Suppose $g = \sum_i h_i$, where $h_i$ is a homogeneous polynomial. Since each $h_i$ component vanishes for any $q \in C(Y)$ (since $h_i \in I(C(Y))$ for each homogenous $h_i$), then this means that $g$ overall also vanishes at $q$, hence $g \in I(C(Y))$. This shows that $f \in I(Y)$ then $f \in I(C(Y))$, thus $I(Y) \subseteq I(C(Y))$.

Second, le't show that $I(C(Y)) \subset I(Y)$. Let $f \in I(C(Y))$. Since $f$ vanishes at all points in $C(Y)$, and $(0, \ldots, 0) \in C(Y)$, then we know $f(0, \ldots, 0) = 0$, so like before, we know that $f$ is a polynomial with \emph{zero} constant term. $f$ is a graded ring over homogeneous polynomials, so we can write:
$$ f = \sum_i h_i,$$
where $h_i$ is a homogeneous polynomial of degree $i$. Note that if a point $q \in C(Y)$ then the entire line is in $C(Y)$, i.e. $\{ \lambda q : \lambda \in k \} \subseteq C(Y)$. Since $f \in I(C(Y))$, then we know that $f(\lambda q) = 0$ for all $\lambda \in k$. This means that:
$$ f(\lambda q) = \sum_i^n h_i(\lambda q) = \sum_i^n \lambda^i h_i(q) = 0, $$
and we can view this as a polynomial $g(\lambda) = f(\lambda q)$ where the $h_i(q)$ terms are seen as coefficients to $\lambda$. Note that:
$$ g(\lambda) =  \sum_i^n \lambda^i h_i(q) = 0, $$
for all $\lambda \in k$ implies that $g$ has infinitely many roots. However, note that by the Fundamental Theorem of Algebra, $g$ is a non-zero polynomial of degree $n$ and must have at most $n$ many roots, which is a contradiction. $g(\lambda) = 0$ for all $\lambda \in k$ if and only if $g$ is the \emph{zero polynomial}. This means that the coefficients of $g$ are all zero, i.e. $h_i(q) = 0$ for all $i$. With this, we've shown that all the homogeneous components of $f$ must vanish at any point $q \in C(Y)$. Since each $h_i$ is a homogeneous polynomial that vanishes at any point $q \in C(Y)$, then this means that for any projective point $[q] \in Y$, $h_i$ vanishes on any of its corresponding coordinates $q$, which implies that $h_i \in I(Y)$. Since $f = \sum_i h_i$, then $f$ also vanishes for any projective point $[q] \in Y$, i.e. $f \in I(Y)$. Finally, we see that $I(C(Y)) \subset I(Y)$ like we wanted to show.

Putting everything together, we see that, indeed, $I(C(Y)) = I(Y)$. \qed

    \item ($\Rightarrow$) Suppose $C(Y)$ is irreducible. From (a), we know that the ideal of $C(Y)$ is $I(Y)$, and since $C(Y)$ is irreducible, then $I(Y)$ must be a prime ideal. From \textbf{1.2.4b}, we know that $I(Y)$ being prime implies that $Y$ itself must be irreducible.

($\Leftarrow$) Suppose $Y$ is irreducible. Then this means that $I(Y)$ is a prime ideal. Furthermore, since $I(Y)$ is the ideal of $C(Y)$, i.e. $I(C(Y)) = I(Y)$, then this also implies that $C(Y)$ itself is irreducible. \qed

\end{enumerate}