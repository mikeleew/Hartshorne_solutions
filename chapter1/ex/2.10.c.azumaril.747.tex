\subsection*{2.10 (c) {azumaril.747}}

First we prove the case when $Y$ is irreducible. 

Let $\alpha_i:\A^n\to \A^{n+1}$ be the embedding defined by $\alpha_i((x_1,\ldots,x_n))=(x_1,\ldots,1,\ldots,x_n)$, where the coordinate component $1$ is on the $i$-th position. 

Let $U_i=\P^n-H_i$ ($i=0,1,\ldots,n$) be the open subsets of $\P^n$ defined in the text, with the homeomorphisms $\varphi_i:U_i\to\A^n$. By Corollary 2.3, we have $Y=\cup_i (Y\cap U_i)$. It follows that $\dim Y=\dim Y\cap U_i$ for some $i$, by Exercise 1.10 (b). Now consider the composition of maps $Y\cap U_i\xrightarrow{\varphi_i}\A^n \xhookrightarrow{\alpha_i}\A^{n+1}$, which sends $(x_0,x_1,\ldots,x_n)\in Y\cap U_i$ to $(\frac{x_o}{x_i},\ldots,1,\ldots,\frac{x_n}{x_i})\in \A^{n+1}$. The composition $\alpha_i\circ\varphi_i$ is an embedding with the image $C(Y)\cap Z(x_i-1)\subseteq \A^{n+1}$. So we have $\dim Y=\dim (C(Y)\cap Z(x_i-1))$. By Exercise 1.8, since $Z(x_i-1)$ is a hypersurface in $\A^{n+1}$, we have $\dim (C(Y)\cap Z(x_i-1))=\dim C(Y)-1$. So $\dim Y=\dim C(Y)-1$. 

When $Y$ is not irreducible, we decompose $Y$ into irreducible components $Y=\cup_i Y_i$. So, by definition, we have $C(Y)=\theta^{-1}(Y)\cup\{(0,\ldots,0)\}=\theta^{-1}(\cup_i Y_i)\cup\{(0,\ldots,0)\}=\cup_i \theta^{-1}(Y_i)\cup\{(0,\ldots,0)\}=\cup_i C(Y_i)$. This gives an irreducible cover of $C(Y)$, by Exercise 2.10 (b). We have $\dim C(Y)=\max \dim C(Y_i)=\max(\dim Y_i+1)=\max(\dim Y_i)+1=\dim Y +1$, so we finish the proof. 

Details of $\alpha_i\circ\varphi_i(Y\cap U_i)=C(Y)\cap Z(x_i-1)$: Let $\a:=I(Y)$ be the ideal of vanishing polynomials on $Y$. Suppose that $\a$ is generated by finitely many homogeneous polynomials $\{f_1,\cdots,f_r\}$. Then $Z^{\P}(\a)=Y$ and $Z^{\A}(\a)=C(Y)$, by Exercise 2.10 (a). We may assume without loss of generality that $i=0$. Then $C(Y)\cap Z(x_0-1)$ is the zero locus of the polynomials $\{f_1(x_0,x_1,\cdots,x_n),\cdots,f_r(x_0,x_1,\cdots,x_n), x_0-1\}$, or equivalently the zero locus of $\{f_1(1,x_1,\cdots,x_n),\cdots,f_r(1,x_1,\cdots,x_n), x_0-1\}$. Notice that $f_i(1,x_1,\cdots,x_n)$ is just $\alpha(f_i)$, and $\varphi_0(Y\cap U_0)$ is exactly $Z(\alpha(f_1),\cdots,\alpha(f_r))\subseteq \A^n$, where $\alpha$ is the notation in the proof of Proposition 2.2. So the claim follows. 

Remark: My intuition is that if we truncate the cone $C(Y)$ by the hyperplane $x_i=1$, the intersection should be isomorphic to $Y\cap U_i$. Cutting by a hyperplane reduces the dimension of $C(Y)$ by $1$.  