\documentclass{article}
\usepackage[utf8]{inputenc}
\usepackage{amsmath,enumitem,amsfonts}
\usepackage{mathtools}
\usepackage{subcaption}
\usepackage{caption}
\usepackage{comment}
\usepackage{float}
\usepackage{biblatex}
\usepackage{amsthm}
\usepackage{amsfonts}
\usepackage{amssymb}
\usepackage{mathrsfs}
\usepackage{enumitem}
\usepackage{tikz}
\usepackage{tikz-cd}
\usepackage[, total={6.5in, 8in}]{geometry}
\usepackage[colorlinks=true, linkcolor=blue, urlcolor=blue]{hyperref}

% \usepackage{PRIMEarxiv}

% \usepackage[T1]{fontenc}    % use 8-bit T1 fonts
\usepackage{hyperref}       % hyperlinks
\usepackage{url}            % simple URL typesetting
\usepackage{booktabs}       % professional-quality tables
\usepackage{nicefrac}       % compact symbols for 1/2, etc.
\usepackage{microtype}      % microtypography
\usepackage{lipsum}
\usepackage{fancyhdr}       % header
\usepackage{graphicx}       % graphics
\usepackage{listings}       % for viewing code
\usepackage[most]{tcolorbox}
\usepackage{tcolorbox}
%\usepackage{upgreek}



\usetikzlibrary{shapes.geometric, arrows.meta, positioning, decorations.markings, calc, fit}

% \newtheorem{lemma}{Lemma}

\definecolor{theoremblue}{HTML}{00008B}
\newcommand{\thm}[2]{
\begin{tcolorbox}[colback=white,colframe=theoremblue!25,title=\textbf{#1}]
#2
\end{tcolorbox}
}

\graphicspath{{media/}}     % organize your images and other figures under media/ folder
\usepackage{tabularray}

\newcommand{\red}[1]{\textcolor[HTML]{b0342c}{#1}}
\newcommand{\floor}[1]{\left\lfloor#1\right\rfloor}
\newcommand{\ceil}[1]{\left\lceil#1\right\rceil}
\renewcommand{\to}{\Rightarrow}
\renewcommand{\tau}{\uptau}

\renewcommand{\P}{\mathbb{P}}  
\let\oldemptyset\emptyset
\let\emptyset\varnothing

\newcommand{\surj}{\twoheadrightarrow}
\newcommand{\inj}{\hookrightarrow}
\newcommand{\trace}{\text{trace}}
\newcommand{\diag}{\text{diag}}
\newcommand{\im}{\text{im}}
\newcommand{\id}{\text{id}}
\newcommand{\sgn}{\text{sgn}}
\newcommand{\abel}{\text{abel}}
\newcommand{\vepo}{\varepsilon_{f(0)} * f }
\newcommand{\vepi}{f * \varepsilon_{f(1)}}

\newcommand{\ad}{\text{ad}}
\newcommand{\Ad}{\text{Ad}}
\newcommand{\Id}{\text{Id}}
\newcommand{\SL}{\text{SL}}
\newcommand{\GL}{\text{GL}}
\newcommand{\SU}{\text{SU}}
\newcommand{\Ab}{\textbf{Ab}}
\newcommand{\Top}{\textbf{Top}}
\newcommand{\Rings}{\textbf{Rings}}
\newcommand{\1}{\mathbf{1}}  % Indicator 1
\newcommand{\rank}{\text{rank}}


\newcommand{\Hom}{\text{Hom}}
\newcommand{\Ext}{\text{Ext}}
\newcommand{\Tor}{\text{Tor}}
\newcommand{\obj}{\text{obj}}
\newcommand{\su}{\mathfrak{su}}
\newcommand{\SO}{\text{SO}}
\newcommand{\so}{\mathfrak{so}}
\newcommand{\Lie}{\text{Lie}}
\newcommand{\RPZ}{\mathbb{R}/2\pi\mathbb{Z}}
\newcommand{\RZ}{\mathbb{R}/\mathbb{Z}}
\newcommand{\SNS}{S^n\!/\!\!\sim}
\newcommand{\HH}{\mathfrak{H}}
\newcommand{\RR}{\mathfrak{R}}
\renewcommand{\d}{\partial}
\newcommand{\bigast}{\mathop{\mbox{\Large $\ast$}}}



\newcommand{\N}{\mathbb{N}}    % Natural numbers
\newcommand{\A}{\mathbb{A}}    % Affine space
\newcommand{\Z}{\mathbb{Z}}    % Integers
\newcommand{\Q}{\mathbb{Q}}    % Rational numbers
\newcommand{\R}{\mathbb{R}}    % Real numbers
\newcommand{\C}{\mathbb{C}}    % Complex numbers
\newcommand{\F}{\mathbb{F}}    % Generic field
\newcommand{\E}{\mathbb{E}}    % Expected value
\newcommand{\RP}{\mathbb{RP}}  % Real projective space RP^n
\newcommand{\CP}{\mathbb{CP}}  % Real projective space CP^n
\newcommand{\simneq}{\nsim}
\newcommand{\vep}{\varepsilon}
\newcommand{\bracket}[1]{\langle #1 \rangle}


\newcommand{\g}{\mathfrak{g}}  % Lie algebra g for Lie group G
\newcommand{\h}{\mathfrak{h}}  % Lie algebra h for Lie group H
\renewcommand{\k}{\mathfrak{k}} % Lie algebra k
\renewcommand{\mod}{\text{mod }} % Lie algebra k
\newcommand{\BB}{\mathcal{B}}  % Bases for spaces
\renewcommand{\AA}{\mathcal{A}}  % Atlas for manifolds
\newcommand{\CC}{\mathcal{C}}  % Category
\renewcommand{\a}{\mathfrak{a}}  % Ideal notation for ideal 'a'


% Operators
\newcommand{\spann}[1]{\operatorname{span}\{#1\}}  % operator: span{X}
\renewcommand{\sl}{\mathfrak{sl}}% Lie algebra of SL
\newcommand{\fundgroup}[1]{\pi_{1}(#1)}
\newcommand{\fundgroupp}[2]{\pi_{1}(#1,#2)}
\newcommand{\normalsub}{\trianglelefteq}
\newcommand{\equivalent}{\Longleftrightarrow}


\let\oldexp\exp    % Save the old definition if you need it later
\renewcommand{\exp}[1][]{\text{exp}\if\relax\detokenize{#1}\relax\else(#1)\fi}
\renewcommand{\to}{\rightarrow}
\newcommand{\coto}{\leftarrow}


%Header
\pagestyle{fancy}
\rhead{ \textit{ }} 

\newenvironment{notation}
  {\begin{center}
   \small\ttfamily
   \begin{minipage}{0.6\textwidth}
   \setlength{\parindent}{0pt}}
  {\end{minipage}
   \end{center}}





\newtheorem{theorem}{Theorem}[section]
\newtheorem{lemma}{Lemma}[theorem]
\newtheorem{corollary}{Corollary}[theorem]
\newtheorem{proposition}{Proposition}[section]
\setlength{\parindent}{0cm}
\setcounter{section}{0}
\setenumerate[0]{label=\alph*)}
\title{Hartshorne Solutions}
\author{mlwells} 
\date{2025}

\begin{document}

\maketitle
Feel free to add your Discord name to the author list.  Let's add our usernames to our solutions, too, so that people get proper credit.

\tableofcontents
\section*{1.1 Affine Varieties}
    \subsection*{1.8}
    \addcontentsline{toc}{subsection}{1.8}
    Let $Y$ be an affine variety of dimension $r$ in $\mathbb{A}^n$.  Let $H$ be a hypersurface in $\mathbb{A}^n$, and assume that $Y \not\subseteq H$.  Then every irreducible component of $Y \cap H$ has dimension $r-1$.


    \subsection*{1.8    \href{https://github.com/mikeleew}{mlwells}
}
    
    Suppose that $H = Z(f)$ with $f$ irreducible in $k[x_1,\ldots, x_n]$.  Then the projection of $f$ in $A(Y) := k[x_1,\ldots,x_n]/I(Y)$ is not equal to $\overline{0}$ since by assumption $(f) \not\subset I(Y)$.  Since $A(Y)$ is a domain (due to the irreducibility of $Y$), the element $\overline{f}$ is not a zero divisor.  Assuming that $Y \cap H \neq \varnothing$, we have that $\overline{f}$ is not a unit in $A(Y)$.  To see this, let $P \in Y \cap H$.  Then $I(Y), (f) \subset \mathfrak{m}_P$, the maximal ideal of $k[x_1,\ldots,x_n]$ corresponding to $P$.  This implies $(\overline{f}) \subset \mathfrak{m}_P / I(Y)$, the latter being a maximal ideal in $A(Y)$.  Thus, $\overline{f}$ is not a unit.

    We apply Theorem 1.11A to get that every minimal prime ideal $\mathfrak{p}$ in $A(Y)$ containing $\overline{f}$ has height 1.  The irreducible components of $Y \cap H$ and the minimal prime ideals containing $(\overline{f})$ correspond, and since each of these prime ideals has height 1, the corresponding varieties have dimension $r-1$ by Theorem 1.8A.

\section*{2.1 Sheaves}
\addcontentsline{toc}{section}{2.1 Sheaves}

\subsection*{1.2}
\addcontentsline{toc}{subsection}{1.2}
\begin{enumerate}
    \item For any morphism of sheaves $\phi: \mathscr{F}\to\mathscr{G}$, show that for each point $P$, $(\ker \phi )_{P}=\ker(\phi_{P})$ and $(\operatorname{im}\phi)_{P} = \operatorname{im}(\phi_{P})$.
    \item Show that $\phi$ is injective (surjective) iff induced map on stalks $\phi_{P}$ is injective (surjective) for all $P$.
    \item Show that the sequence of sheaves $\begin{tikzcd}[cramped, sep=small] \cdots \arrow[r] & \mathscr{F}^{i-1} \arrow[r] & \mathscr{F}^{i} \arrow[r] & \mathscr{F}^{i+1} \arrow[r] & \cdots \end{tikzcd}$ is exact iff coresponding sequences of stalks are exact for all $P$.

\end{enumerate}
    \subsection*{1.2   \href{https://github.com/yakimk}{yakimk}
}

\begin{enumerate}
    \item Since stalks are defined as a filtered colimit and kernels are a particular type of limit (pullback) and \href{https://stacks.math.columbia.edu/tag/002W}{filtered colimits commute with finite limits} isomorphism $(\ker \phi_{P}) = \ker(\phi_{P})$ is automatic.

To show a similar thing for the image we first note that $(\operatorname{im}\phi)_{P} \simeq ({(\operatorname{im}_{pre}\phi)^{+})_{P}}\simeq (\operatorname{im}_{pre} \phi)_{P}$, where the first isomorphism is by definition and the second one using the fact that sheafification has the same stalks as original presheaf.  Now we conclude noticing that $\operatorname{im}_{pre}\phi \simeq \ker(\operatorname{coker}\phi)$, i.e. it is a finite limit of finite colimits and since again stalks are filtered colimits they commute with finite limits and colimits (in for instance abelian category) we conclude that $(\operatorname{im} \phi)_{P} \simeq \operatorname{im}(\phi_{P})$.

\item We show that a map of sheaves $\phi: \mathscr{F} \to \mathscr{G}$ is injective iff it induced maps on stalks are injective (proof for surjectivity is essentially identical).

Consider the following commutative diagram
\[
\begin{tikzcd}
\mathscr{F}(U)     \arrow[r,"\phi_{U}"] \arrow[d,"\iota _{1}"'] &  \mathscr{G}(U) \arrow[d,"\iota_{2}"]\\ 
\prod_{P \in U} \mathscr{F}_{p}     \arrow[r,"\prod \tilde{\phi }_{P}"] &  \prod_{P \in U}\mathscr{G}_{p}
\end{tikzcd}
\]

Note that since section of a sheaf on an open set is determined by its stalks, $\iota_{1}$ and $\iota_{2}$ are both injective.

Suppose $\tilde{\phi}_{P}$ is injective for all $P$. Take two sections $f, g \in \mathscr{F}(U)$, since $\prod \tilde{\phi}_{P}$ and $\iota_{1}$ are injective (former by the hypothesis that $\tilde{\phi}_{P}$ are injective), their composition is injective. If $\phi_{U}$ were not injective, we could find two different sections that go to the same class in $\prod \mathscr{G}_{P}$, which would contradict injectivity of $\prod \tilde{\phi}_{P} \circ \iota_{1}$.

Similarly if $\phi_{U}$ is injective then if some of $\tilde{\phi}_{P}$  were not injective, its composition with $\iota_{1}$ would not be injective contradicting commutativity of the diagram above. 

\item We have to show that 
$\begin{tikzcd}[cramped, sep=small] \cdots \arrow[r] & \mathscr{F}^{i-1} \arrow[r] & \mathscr{F}^{i} \arrow[r] & \mathscr{F}^{i+1} \arrow[r] & \cdots \end{tikzcd}$
is exact iff induced sequence of stalks are exact for all $P$.

Assuming that sequence of sheaves is exact we show induced sequences are exact at all stalks. By the previuos section $\operatorname{im} \phi_{P}^i \simeq (\operatorname{im} \phi^i)_{P}\simeq(\ker \phi^{i+1})_{P}\simeq \ker \phi_{P}^{i + 1}$.

Now assume that all sequences of stalks are exact.  By the same reasoning as before we see that  $\operatorname{im} \phi^{i}\simeq \ker \phi^{i+1}$ (since they are isomorphic on each stalk and hence are equal).
\end{enumerate}


\end{document}