\documentclass{article}
\usepackage[utf8]{inputenc}
\usepackage{amsmath,enumitem,amsfonts}
\usepackage{mathtools}
\usepackage{subcaption}
\usepackage{caption}
\usepackage{comment}
\usepackage{float}
\usepackage{biblatex}
\usepackage{amsthm}
\usepackage{amsfonts}
\usepackage{amssymb}
\usepackage{mathrsfs}
\usepackage{enumitem}
\usepackage{tikz}
\usepackage{tikz-cd}
\usepackage[, total={6.5in, 8in}]{geometry}
\usepackage[colorlinks=true, linkcolor=blue, urlcolor=blue]{hyperref}

% \usepackage{PRIMEarxiv}

% \usepackage[T1]{fontenc}    % use 8-bit T1 fonts
\usepackage{hyperref}       % hyperlinks
\usepackage{url}            % simple URL typesetting
\usepackage{booktabs}       % professional-quality tables
\usepackage{nicefrac}       % compact symbols for 1/2, etc.
\usepackage{microtype}      % microtypography
\usepackage{lipsum}
\usepackage{fancyhdr}       % header
\usepackage{graphicx}       % graphics
\usepackage{listings}       % for viewing code
\usepackage[most]{tcolorbox}
\usepackage{tcolorbox}
%\usepackage{upgreek}



\usetikzlibrary{shapes.geometric, arrows.meta, positioning, decorations.markings, calc, fit}

% \newtheorem{lemma}{Lemma}

\definecolor{theoremblue}{HTML}{00008B}
\newcommand{\thm}[2]{
\begin{tcolorbox}[colback=white,colframe=theoremblue!25,title=\textbf{#1}]
#2
\end{tcolorbox}
}

\graphicspath{{media/}}     % organize your images and other figures under media/ folder
\usepackage{tabularray}

\newcommand{\red}[1]{\textcolor[HTML]{b0342c}{#1}}
\newcommand{\floor}[1]{\left\lfloor#1\right\rfloor}
\newcommand{\ceil}[1]{\left\lceil#1\right\rceil}
\renewcommand{\to}{\Rightarrow}
\renewcommand{\tau}{\uptau}

\renewcommand{\P}{\mathbb{P}}  
\let\oldemptyset\emptyset
\let\emptyset\varnothing

\newcommand{\surj}{\twoheadrightarrow}
\newcommand{\inj}{\hookrightarrow}
\newcommand{\trace}{\text{trace}}
\newcommand{\diag}{\text{diag}}
\newcommand{\im}{\text{im}}
\newcommand{\id}{\text{id}}
\newcommand{\sgn}{\text{sgn}}
\newcommand{\abel}{\text{abel}}
\newcommand{\vepo}{\varepsilon_{f(0)} * f }
\newcommand{\vepi}{f * \varepsilon_{f(1)}}

\newcommand{\ad}{\text{ad}}
\newcommand{\Ad}{\text{Ad}}
\newcommand{\Id}{\text{Id}}
\newcommand{\SL}{\text{SL}}
\newcommand{\GL}{\text{GL}}
\newcommand{\SU}{\text{SU}}
\newcommand{\Ab}{\textbf{Ab}}
\newcommand{\Top}{\textbf{Top}}
\newcommand{\Rings}{\textbf{Rings}}
\newcommand{\1}{\mathbf{1}}  % Indicator 1
\newcommand{\rank}{\text{rank}}


\newcommand{\Hom}{\text{Hom}}
\newcommand{\Ext}{\text{Ext}}
\newcommand{\Tor}{\text{Tor}}
\newcommand{\obj}{\text{obj}}
\newcommand{\su}{\mathfrak{su}}
\newcommand{\SO}{\text{SO}}
\newcommand{\so}{\mathfrak{so}}
\newcommand{\Lie}{\text{Lie}}
\newcommand{\RPZ}{\mathbb{R}/2\pi\mathbb{Z}}
\newcommand{\RZ}{\mathbb{R}/\mathbb{Z}}
\newcommand{\SNS}{S^n\!/\!\!\sim}
\newcommand{\HH}{\mathfrak{H}}
\newcommand{\RR}{\mathfrak{R}}
\renewcommand{\d}{\partial}
\newcommand{\bigast}{\mathop{\mbox{\Large $\ast$}}}



\newcommand{\N}{\mathbb{N}}    % Natural numbers
\newcommand{\A}{\mathbb{A}}    % Affine space
\newcommand{\Z}{\mathbb{Z}}    % Integers
\newcommand{\Q}{\mathbb{Q}}    % Rational numbers
\newcommand{\R}{\mathbb{R}}    % Real numbers
\newcommand{\C}{\mathbb{C}}    % Complex numbers
\newcommand{\F}{\mathbb{F}}    % Generic field
\newcommand{\E}{\mathbb{E}}    % Expected value
\newcommand{\RP}{\mathbb{RP}}  % Real projective space RP^n
\newcommand{\CP}{\mathbb{CP}}  % Real projective space CP^n
\newcommand{\simneq}{\nsim}
\newcommand{\vep}{\varepsilon}
\newcommand{\bracket}[1]{\langle #1 \rangle}


\newcommand{\g}{\mathfrak{g}}  % Lie algebra g for Lie group G
\newcommand{\h}{\mathfrak{h}}  % Lie algebra h for Lie group H
\renewcommand{\k}{\mathfrak{k}} % Lie algebra k
\renewcommand{\mod}{\text{mod }} % Lie algebra k
\newcommand{\BB}{\mathcal{B}}  % Bases for spaces
\renewcommand{\AA}{\mathcal{A}}  % Atlas for manifolds
\newcommand{\CC}{\mathcal{C}}  % Category
\renewcommand{\a}{\mathfrak{a}}  % Ideal notation for ideal 'a'


% Operators
\newcommand{\spann}[1]{\operatorname{span}\{#1\}}  % operator: span{X}
\renewcommand{\sl}{\mathfrak{sl}}% Lie algebra of SL
\newcommand{\fundgroup}[1]{\pi_{1}(#1)}
\newcommand{\fundgroupp}[2]{\pi_{1}(#1,#2)}
\newcommand{\normalsub}{\trianglelefteq}
\newcommand{\equivalent}{\Longleftrightarrow}


\let\oldexp\exp    % Save the old definition if you need it later
\renewcommand{\exp}[1][]{\text{exp}\if\relax\detokenize{#1}\relax\else(#1)\fi}
\renewcommand{\to}{\rightarrow}
\newcommand{\coto}{\leftarrow}


%Header
\pagestyle{fancy}
\rhead{ \textit{ }} 

\newenvironment{notation}
  {\begin{center}
   \small\ttfamily
   \begin{minipage}{0.6\textwidth}
   \setlength{\parindent}{0pt}}
  {\end{minipage}
   \end{center}}





\newtheorem{theorem}{Theorem}[section]
\newtheorem{lemma}{Lemma}[theorem]
\newtheorem{corollary}{Corollary}[theorem]
\newtheorem{proposition}{Proposition}[section]
\setlength{\parindent}{0cm}
\setcounter{section}{0}
\setenumerate[0]{label=\alph*)}
\title{Hartshorne Solutions}
\author{mlwells} 
\date{2025}

\begin{document}

\maketitle
Feel free to add your Discord name to the author list.  Let's add our usernames to our solutions, too, so that people get proper credit.

\tableofcontents
\section*{1.1 Affine Varieties}
\addcontentsline{toc}{section}{1.1 Affine Varieties}

    \subsection*{1.1}
    \addcontentsline{toc}{subsection}{1.1}

    \begin{enumerate}
        \item Let $Y$ be the plane curve $y = x^2$ (i.e., $Y$ is the zero set of the polynomial $f = y - x^2$). Show that $A(Y)$ is isomorphic to a polynomial ring in one variable over $k$.

        \item  Let $Z$ be the plane curve $xy = 1$. Show that $A(Z)$ is not isomorphic to a polynomial ring in one variable over $k$.

        \item Let $f$ be any irreducible quadratic polynomial in $k[x,y]$, and let $W$ be the conic defined by $f$. Show that $A(W)$ is isomorphic to $A(Y)$ or $A(Z)$. Which one is it when? \red{[Optional]}


    \end{enumerate}
    \subsection*{1.1 (a)   {edealba}}
    
    \hspace{1cm} Note that $A(Y)$ is the coordinate ring for the curve $y-x^2 = 0$, and since we're working over the plane, suppose we're working on $\A^2$ (affine plane) over the field $k$, such that the coordinate ring of the plane is just $k[x, y]$. We want to show that $A(Y) \cong k[x]$.

Note that the coordinate ring $A(Y)$ is defined as the following quotient ring:
$$ A(Y) = k[x, y]/J, $$
where $J = (y-x^2)$, i.e. $J$ is the ideal generated by $y-x^2$. Consider the following mapping:
$$ \phi: k[x, y] \to k[x], $$
where $\phi$ is an evaluation mapping on $y$, where $y \mapsto x^2$. Evaluation maps like $\phi$ are ring homomorphisms. We will show that $\ker(\phi)$ is exactly equivalent to $J$ since all polynomials generated by $y-x^2$ get mapped to $x^2-x^2 = 0$ via $\phi$, hence $J \subset \ker(\phi)$. To show equality, we need to show that $\ker(\phi) \subset J$. Let $a \in \ker(\phi)$, i.e. $\phi(a) = 0$. We want to show that $a \in J$, i.e. that we can write $a$ as:
$$a = A(x, y) \cdot (y-x^2), $$
where $A(x, y) \in k[x, y]$. If $a$ were \emph{not} in $J$ then we would have some non-zero remainder $r(x, y)$ with degree of $y$ less than $1$ (since the degree of $y$ in $y-x^2$ is $1$), so we can write the remainder as $r(x)$ such that:
$$a = A(x, y) \cdot (y-x^2) + r(x). $$
Let's apply $\phi$ to the RHS: $\phi(A(x, y) \cdot (y-x^2) + r(x)) = A(x, x^2)\cdot (x^2-x^2) + r(x) = A(x, x^2)\cdot 0 + r(x) = r(x)$. Note that this implies that $\phi(a) = r(x) = 0$. This means that the remainder is zero, and hence $a = A(x, y)\cdot(y-x^2)$ which means that $a \in J$ like we wanted to show. Thus, $\ker(\phi) = J$.

Next, let's show that $\phi$ is surjective. Let $p(x) \in k[x]$ be an arbitary polynomial of degree $m$, so we can write:
$$ p(x) = \sum_{i=0}^{m} c_ix^{i} $$
Note that $c_i \in k \subset k[x] \subset k[x, y]$ and also $x^i \in k[x] \subset k[x, y]$ for all $i$, then clearly $p(x) \in k[x, y]$ where $\phi(p(x)) = p(x)$. Since $\phi$ is surjective, then we know that $\im(\phi) = k[x]$. By the first isomorphism theorem, we get:
$$ k[x, y] / \ker(\phi) \cong \im(\phi), $$
where $\ker(\phi) = J$ and $\im(\phi) = k[x]$, thus
$$ k[x, y] / J \cong k[x],$$
and finally since we know $A(Y) = k[x, y] / J$, then clearly we get that $A(Y) \cong k[x]$, so we've shown that the coordinate ring $A(Y)$ is, indeed, isomorphic to the polynomial rings in one variable over $k$. \qed

  \subsection*{1.1 (b)   {edealba}}

Note that $A(Z)$ is the following:
$$ A(Z) = \Bigg\{ \sum_{i, j \geq 0} c_{ij}\bar{x}^i\bar{y}^j \mid c_{ij} \in k, \text{ only a finite number of $c_{ij}$ are non-zero}, \text{ and } \bar{x}\bar{y}=1  \Bigg\} $$

Suppose there were a ring isomorphism $\phi: A(Z) \to k[t]$, where we choose $t$ to be the polynomial ring of one variable over $k$. Note that ring isomorphisms must map units to units. Since we're working over the plane curve $xy = 1$, this means that the image of $x$ in $A(Z)$ is unit, i.e. $\bar{x} \in A(Z)$ is unit. Note that the units in $k[t]$ are all the non-zero constants, i.e. $c \in k^{\times} \subset k[t]$, where these are just the non-zero elements of the field $k$. Since $\phi$ maps units to units, then $\bar{x}$ must get mapped to some non-zero constant in $k[t]$. Note that if $\bar{x}$ gets mapped to a constant, then $\im(\phi) = k$. This is because if we have some arbitary $p \in A(Z)$ then we have:
$$ p = \sum_{i, j \geq 0} c_{ij}\bar{x}^i\bar{y}^j, $$
but note that if $\bar{x}\mapsto c_x$ where $c_x$ is some non-zero constant in $k$ then this implies that $\bar{y}\mapsto c_y$ where $c_y$ is also some non-zero constant such that $c_y = 1/c_x$. This means that $\phi(p)$ is:
$$ \phi(p) = \sum_{i, j \geq 0} c_{ij}c_x^{i} c_y^{j}, $$
which is just some constant element of $k$ for any $p \in A(Z)$. Note that $\im(\phi) = k$ is a contradiction since we assumed $\phi$ was an isomorphism, i.e. $\im(\phi) = k[t]$. This means there is no such ring isomorphism $\phi: A(Z) \to k[t]$. \qed

\subsection*{1.3}
\addcontentsline{toc}{subsection}{1.3}

Let $Y$ be the algebraic set in $\mathbb{A}^3$ defined by the two polynomials $x^2 - yz = 0$ and $xz - x = 0$.
Show that $Y$ is a union of three irreducible components.  Describe them and find their prime ideals.
\subsection*{1.3   {azumaril.747}}

We have $Y=Z(x^2-yz,xz-x)=Z(x^2-yz)\cap Z(x(z-1))=Z(x^2-yz)\cap(Z(x)\cup Z(z-1))=(Z(x^2-yz)\cap Z(x))\cup (Z(x^2-yz)\cap Z(z-1))$. The first part $Z(x^2-yz)\cap Z(x)=Z(x^2-yz,x)=Z(yz,x)=Z(x,y)\cup Z(z,x)$. The second part $Z(x^2-yz)\cap Z(z-1)=Z(x^2-y,z-1)$. So the three irreducible components are $Z(x,y)$, $Z(x,z)$, and  $Z(x^2-y,z-1)$, with prime ideals $(x,y)$, $(x,z)$, and $(x^2-y,z-1)$, respectively. Geometrically, they are two affine lines and one parabolic curve. 

    \subsection*{1.4}
    \addcontentsline{toc}{subsection}{1.4}

    If we identify $\mathbb{A}^2$ with $\mathbb{A}^1 \times \mathbb{A}^1$ in the natural way, show that the Zariski topology on $\mathbb{A}^2$ is not the product topology of the Zariski topologies on the two copies of $\mathbb{A}^1$.

  \subsection*{1.4   {edealba}}

  Consider the diagonal line $y=x$ in $\A^2$ where we're working over the coordinate ring $k[x, y]$. Note that $V(y-x) \subset \A^2$ is closed in the Zariski topology of $\A^2$, and is the following:
$$ V(y-x) = \left\{ p \in k^2 : f(p) = 0 \text{ for all } f \in (y-x) \right\} $$
However, when we look at the closed sets in $\A^1$, we only have the following: (1) sets with a finite number of points, (2) the entire affine line $\A^1$. In $\A^1 \times \A^1$ we have: (1) also sets with finite number of points, (2) the entire space $\A^1 \times \A^1$, and (3) we also finite unions of vertical and horizontal lines from $\A^1 \times \{ p_i \}$ and $\{ p_j \} \times \A^1$ for points $p_i$ and $p_j$ in $\A^1$. Note that it's \emph{impossible} to construct the diagonal line $y-x=0$ using any combination of closed sets in $\A^1 \times \A^1$, we would need an infinite number of points which is \emph{not} allowed in the Zariski topology. Thus, although $\A^2$ and $\A^1 \times \A^1$ may be identical to each other, their topologies are not the same. \qed


    \subsection*{1.5}
    \addcontentsline{toc}{subsection}{1.5}

    Show that a $k$-algebra $B$ is isomorphic to the affine coordinate ring of some algebraic set in $\mathbb{A}^n$, for some $n$, if and only if $B$ is a finitely generated $k$-algebra with no nilpotent elements.


  \subsection*{1.5   {edealba}}

  Let $B$ be a $k$-algebra, and let $W = V(J) \subset \A^n$ be some algebraic set of $\A^n$ for some ideal $J \subset k[x_1, \ldots, x_n]$, where $W$ is the set of all points that vanish for all polynomial functions in $J$. We then have the following coordinate ring (which is the following quotient ring):
$$ A(W) = k[x_1, \ldots, x_n] / I(W),$$
and since $W = V(J)$, by Hilbert's strong Nullstellensatz, we know that $I(V(J)) = \sqrt{J}$, hence
$$ A(W) = k[x_1, \ldots, x_n] / \sqrt{J} $$
We want to show that $B \cong A(W)$ iff $B$ is a finitely generated $k$-algebra with no nilpotent elements.

($\Rightarrow$) First, suppose $B \cong A(W)$. Since we've defined $A(W)$ to be the quotient ring of $k[x_1, \ldots, x_n]$ modded out by the radical of an ideal $\sqrt{J}$, then this is automatically a finitely generated $k$-algebra. We want to show that it has no nilpotent elements, i.e. there is \emph{no} non-zero element $f \in B \cong A(W)$ such that $f^m = 0$ for some $m \geq 2$.

Suppose for contradiction that there exists some nilpotent element $f \in B \cong A(W)$ such that $f^m = 0$ for some $m \geq 2$. Since $f$ is assumed to be non-zero, this means that $f \notin I(W) = \sqrt{J}$. Since $J \subset \sqrt{J}$, then $f \notin \sqrt{J}$ implies that $f \notin J$. Note that since $f \notin I(W)$ then this also means that there exists some point $p' \in W$ such that $f(p') \neq 0$. Next, since we have $f^m = 0$, then $f^m \in I(W) = \sqrt{J}$. This means that $f^m$ vanishes for all points $p \in W$, and hence $f^m(p') = 0$. Since $f^m(p') = 0$, this implies that $f^m(p') = (f(p'))^m = 0$ which implies that $f(p') =0$ which contradicts our previous $f(p') \neq 0$. With this, we've shown that there \emph{can't} be a nilpotent element $f \in B \cong A(W)$.

($\Leftarrow$) Let $B$ be a finitely generated $k$-algebra with generators $\{ t_1, \ldots, t_n \} \subset B$ such that any element in $B$ can be written as a polynomial in these generators, and $B$ also has no nilpotent elements. Consider the following mapping:
$$ \phi: k[x_1, \ldots, x_n] \to B, $$
where $\phi$ maps $x_i \mapsto t_i$ for all $1\leq i \leq n$. This mapping is clearly surjective since any polynomial $f \in B$ can be written by swapping each $t_i$ with its corresponding $x_i$. Note that if the generators $\{t_1, \ldots, t_n \}$ have some algebraic relations, this is captured by $\ker(\phi)$. Let $J = \ker(\phi) \subset k[x_1, \ldots, x_n]$ be the ideal generated by all the algebraic relations between the generators $\{ t_1, \ldots, t_n \}$. By the First isomorphism theorem, we get:
$$ k[x_1, \ldots, x_n]/J \cong B$$
Next, we want to show that $J$ is a radical ideal. Note that there are \emph{no} nilpotent elements in $B$. If there \emph{were} some nilpotent element then suppose $f \in B$ were nilpotent, then this would mean that $f^m = 0$ for some $m\geq 1$.

(1) An ideal $J$ is radical if for some $x^m \in J$ with $m \geq 1$ implies that $x \in J$, i.e.:
$$ x^m \in J \implies x \in J  $$

(2) Since $B \cong k[x_1, \ldots, x_n]/J$, then elements of $B$ are of the coset form $\bar{f} = f + J$. If $\bar{f} \in B$ were nilpotent then this means that $\bar{f}^m = 0$ for some $m\geq 1$, which means that $f^m + J = 0$ which implies that $f^m \in J$ for $f\in k[x_1, \ldots, x_n]$.

Note that since there are \emph{no} nilpotent elements in $B$, then this means for all non-zero $\bar{f} \in B$ we know that $\bar{f} = f + J \neq 0$, i.e. $f \notin J$ where $f^m \neq 0$ for all $m\geq 1$. Furthermore, note that the contrapositive of (1) means that if the following holds:
$$ x \notin J \implies x^m \notin J \text{ for all } m\geq 1, $$
then $J$ is a radical ideal. We know that for any non-zero $\bar{f}$ we have $f \notin J$, and since $\bar{f}$ is \emph{not} nilpotent, then $f^m \notin J$ for all $m\geq 1$. With this, we've shown that $B$ not having any nilpotent elements implies that $J$ is a radical ideal, i.e. $J = \sqrt{J}$.

Finally, let's consider the algebraic set $W = V(J) \subset \A^n$. From Hilbert's strong Nullstellensatz, we know that the coordinate ring $A(W)$ is the following:
$$ A(W) \cong k[x_1, \ldots, x_n]/I(V(J)) = k[x_1, \ldots, x_n]/\sqrt{J}, $$
but note that since $J$ is radical, then we get:
$$ A(W) \cong k[x_1, \ldots, x_n]/J, $$
so we get that $B \cong k[x_1, \ldots, x_n]/J \cong A(W)$ like we wanted to show. \qed


    \subsection*{1.8}
    Let $Y$ be an affine variety of dimension $r$ in $\mathbb{A}^n$.  Let $H$ be a hypersurface in $\mathbb{A}^n$, and assume that $Y \not\subseteq H$.  Then every irreducible component of $Y \cap H$ has dimension $r-1$.

    \subsection*{1.8    \href{https://github.com/mikeleew}{mlwells}
}
    
    Suppose that $H = Z(f)$ with $f$ irreducible in $k[x_1,\ldots, x_n]$.  Then the projection of $f$ in $A(Y) := k[x_1,\ldots,x_n]/I(Y)$ is not equal to $\overline{0}$ since by assumption $(f) \not\subset I(Y)$.  Since $A(Y)$ is a domain (due to the irreducibility of $Y$), the element $\overline{f}$ is not a zero divisor.  Assuming that $Y \cap H \neq \varnothing$, we have that $\overline{f}$ is not a unit in $A(Y)$.  To see this, let $P \in Y \cap H$.  Then $I(Y), (f) \subset \mathfrak{m}_P$, the maximal ideal of $k[x_1,\ldots,x_n]$ corresponding to $P$.  This implies $(\overline{f}) \subset \mathfrak{m}_P / I(Y)$, the latter being a maximal ideal in $A(Y)$.  Thus, $\overline{f}$ is not a unit.

    We apply Theorem 1.11A to get that every minimal prime ideal $\mathfrak{p}$ in $A(Y)$ containing $\overline{f}$ has height 1.  The irreducible components of $Y \cap H$ and the minimal prime ideals containing $(\overline{f})$ correspond, and since each of these prime ideals has height 1, the corresponding varieties have dimension $r-1$ by Theorem 1.8A.


\section*{1.2 Projective Varieties}
\addcontentsline{toc}{section}{1.2 Projective Varieties}
\subsection*{2.1}
\addcontentsline{toc}{subsection}{2.1}

Prove the ``homogeneous Nullstellensatz,'' which says if $J \subseteq S$ is a homogeneous ideal, and if $f \in S$ is a homogeneous polynomial with $\deg f > 0$, such that $f(P) = 0$ for all $P \in Z(J)$ in $\P^n$, then $f^q \in J$ for some $q > 0$. [\textit{Hint:} Interpret the problem in terms of the affine $(n + 1)$-space whose affine coordinate ring is $S$, and use the usual Nullstellensatz, (1.3A).]

  \subsection*{2.1   {edealba}}

Let $W = Z(J) \subset \P^n$ be a projective variety, and note that $f(P) = 0$ means that the homogeneous polynomial $f$ vanishes at each point in $W \subset \P^n$ which represents a line going through the origin in $\A^{n+1}$. Consider the \emph{affine cone} over $W = Z(J)$, which we denote $C(W) \subset \A^{n+1}$. Since $f(P) = 0$ for points $P$ that represent lines going through the origin in $\A^{n+1}$, hence $f(p) = 0$ for all points $p \in C(W) \subset \A^{n+1}$ since these are all points on the lines projected in $W = Z(J) \subset \P^n$.

We want to show that $C(W) = C(Z(J)) = V(J)$, where $V(J) \subset \A^{n+1}$ is the variety of vanishing points of the ideal $J$.

First, let's show that $C(Z(J)) \subset V(J)$. Let $p = (a_0, \ldots, a_n) \in C(Z(J))$ be a point. We have the following cases:

- If $p = (0, \ldots, 0)$, i.e. this is the origin in $\A^{n+1}$, then for any homogeneous polynomial $g \in J$, we get $g(p) = 0$ since homogeneous polynomials don't have a constant term.

- Suppose $p = (a_0, \ldots, a_n) \neq 0$. This is a point in the affine cone, so it's a representative for a point $P \in Z(J) \subset \P^n$ such that $P = [a_0:\cdots:a_n]$. Note that any homogeneous polynomial $g \in J$ vanishes at $P$, by definition of $Z(J)$. Note that $g(P) = 0$ implies that $g(p) = 0$ since $g$ vanishes at any point on the line corresponding to $P$, i.e. for any point $p_{\lambda} = (\lambda a_0, \ldots, \lambda a_n)$ we get:
$$ g(p_{\lambda}) = g(\lambda a_0, \ldots, \lambda a_n) = \lambda^d \cdot g(a_0, \ldots, a_n) =0, $$
where $g$ is a degree $d$ homogeneous polynomial. What about non-homogeneous polynomials in $J$? Let $h$ be any polynomial in $J$. Since $J$ is a homogeneous ideal, then we can write:
$$ h = h_1 + h_2 + \cdots + h_d, $$
where each $h_i$ is a homogeneous polynomial in $J$. Since $h_i(p) = 0$ for each homogeneous $h_i \in J$, then:
$$ h(p) = \sum h_i(p) = 0, $$
which means that any polynomial $h \in J$ vanishes at $p$.

This shows that $p \in C(Z(J))$ implies that $p \in V(J)$, thus
$$ C(W) = C(Z(J)) \subset V(J). $$

Next, let's show that $V(J) \subset C(Z(J))$. Let $p = (a_0, \ldots, a_n) \in V(J)$. This means that any polynomial $g \in J$ vanishes at $p$, i.e. $g(p) = 0$. This holds, in particular, for \emph{homogeneous} polynomials in $J$. Note that this is precisely the condition for the projective point $P = [a_0: a_1: \cdots : a_n]$ to be in $Z(J)$. Since $P \in Z(J)$, then the point $p$ lies on the line representing $P$. By definition of the affine cone, this means that $p \in C(Z(J))$. Thus,
$$ V(J) \subset C(Z(J)). $$

With this, we've shown that $C(W) = C(Z(J)) = V(J)$. Since $V(J) \subset \A^{n+1}$, we can apply Hilbert's Nullstellensatz. Recall that $f$ is a polynomial such that $f(P) = 0$ for all points $P \in Z(J)$. This implies that $f$ vanishes at all points in the affine cone $p \in C(Z(J)) = V(J) \subset \A^{n+1}$. By Hilbert's Nullstellensatz, since $f$ vanishes at all points in $V(J)$, then $f \in \sqrt{J}$ which means that $f^q \in J$ for some $q > 0$ like we wanted to show. \qed

\subsection*{2.3}
\addcontentsline{toc}{subsection}{2.3}

\begin{enumerate}
    \item If $T_1 \subseteq T_2$ are subsets of $S^h$, then $Z(T_1) \supseteq Z(T_2)$.
    \item If $Y_1 \subseteq Y_2$ are subsets of $\P^n$, then $I(Y_1) \supseteq I(Y_2)$.
    \item For any two subsets $Y_1, Y_2$ of $\P^n$, $I(Y_1 \cup Y_2) = I(Y_1) \cap I(Y_2)$.
    \item If $J \subset S$ is a homogenous ideal with $Z(J) \neq \emptyset$, then $I(Z(J)) = \sqrt{J}$.
    \item For any subset $Y \subseteq \P^n$, $Z(I(Y)) = \overline{Y}$.
\end{enumerate}
\subsection*{2.3 (a) (b) {edealba}}

\begin{enumerate}
    \item This follows since $Z$ is a contravaiant functor (or operator), so when we apply $Z$ to $T_1 \subseteq T_2$, we reverse the ``arrows'', in this case we get $\supseteq$, and thus:
$$ Z(T_1) \supseteq Z(T_2). $$
To elaborate, let's consider $T_1 \subseteq T_2 \subseteq S^h$. Let $p \in Z(T_1)$, i.e. $p$ is a point such that every homogeneous polynomial in $T_1$ vanishes at $p$. Does this imply that $p \in Z(T_2)$? Not necessarily. Suppose there exists some polynomial $g \in T_2 - T_1$, where $g \in T_2$ and $g \notin T_1$. Note that $p \in Z(T_1)$ does \emph{not} guarantee that $g(p) = 0$ since $g \notin T_1$. On the other hand, suppose $p' \in Z(T_2)$. This means that every polynomial $f$ in $T_2$ is such that $f(p') = 0$. Since $T_1 \subseteq T_2$, then this still holds for every polynomial in $T_1$, thus $p' \in Z(T_2)$ implies that $p' \in Z(T_1)$, i.e.
$$ Z(T_2) \subseteq Z(T_1), \text{ or } Z(T_1) \supseteq Z(T_2), $$
like we wanted to show. \qed
    \item  If $Y_1 \subseteq Y_2$ are subsets of $\P^n$, then $I(Y_1) \supseteq I(Y_2)$.
Similar to (a), $I$ is also contravaiant functor (or operator), so when we apply $I$ to $Y_1 \subseteq Y_2$, we get:
$$ I(Y_1) \supseteq I(Y_2) $$ \qed
\end{enumerate}
\subsection*{2.4}
\addcontentsline{toc}{subsection}{2.4}

\begin{enumerate}
    \item There is a 1-1 inclusion-reversing correspondence between algebraic sets in $\P^n$, and homogeneous radical ideals of $S$ not equal to $S_+$, given by $Y \mapsto I(Y)$ and $J \mapsto Z(J)$. [\emph{Note:} Since $S_+$ does not occur in this correspondence, it is sometimes called the \emph{irrelevant} maximal ideal of $S$.]
    \item An algebraic set $Y \subseteq \P^n$ is irreducible if and only if $I(Y)$ is a prime ideal.
    \item Show that $\P^n$ itself is irreducible.
\end{enumerate}
\subsection*{2.4  {edealba}}

\begin{enumerate}
    \item Let $J \subset S$ be a homogeneous radical ideal that's \emph{not} $S_+$. We want to show that $I(Z(J)) = J$. Note that $Z(S_+) = \emptyset$. Suppose it were the case that $Z(J) = \emptyset$. This would imply that either $J = S$ or $J$ contains $S_+$. Since $J$ is a proper ideal, by assumption, then $J \neq S$. Furthermore, note that $S_+$ is a maximal \emph{homogeneous} ideal, hence $S_+ \subseteq J$ implies $J = S_+$, but recall that we assumed that $J \neq S_+$. We've reached a contradiction, thus $Z(J) \neq \emptyset$. By \textbf{1.2.3d}, we know that if $\a \subset S$ is a homogenous ideal with $Z(\a) \neq \emptyset$, then $I(Z(\a)) = \sqrt{\a}$. In our case, we know that $J$ is radical by assumption, thus:
$$ I(Z(J)) = \sqrt{J} = J. $$

Next, let $Y \subset \P^n$ be an algebraic set. From \textbf{1.2.3e}, we know that for any subset $W \subseteq \P^n$, we get $Z(I(W)) = \overline{W}$. Since $Y$ is an algebraic set, this means that $\overline{Y} = Y$, hence:
$$ Z(I(Y)) = \overline{Y} = Y, $$
like we wanted to show. \qed
    \item ($\Leftarrow$) Suppose $I(Y)$ is a prime ideal. We want to show that $Y \subseteq \P^n$ is irreducible. Suppose, for contradiction, that $Y$ is reducible, i.e. we can write $Y = C_1 \cup C_2$ where $C_1, C_2 \subsetneq Y$. Since $I(Y)$ is a prime ideal, this means that $f_1f_2 \in I(Y)$ implies that $f_1 \in I(Y)$ or $f_2 \in I(Y)$. Note that since $Y = C_1 \cup C_2$, then we have:
$$ I(Y) = I(C_1 \cup C_2) = I(C_1) \cap I(C_2). $$

Since we have $C_1 \subsetneq Y$, then we have:
$$ I(C_1) \supsetneq I(Y), $$
i.e. $I(Y) \subsetneq I(C_1)$. Let $f_1 \in I(C_1) \setminus I(Y)$. Since $C_1$ is a strictly contained in $Y$, this means that $I(Y)$ is strcitly contained in $I(C_1)$, i.e. $ I(C_1) \setminus I(Y) \neq \emptyset$. Similarly for $C_2 \subsetneq Y$, we get $I(C_2) \setminus I(Y) \neq \emptyset$, so let $f_2 \in I(C_2) \setminus I(Y)$. Note that since $I(C_1) \cap I(C_2) = I(Y)$, then clearly we have that $f_1, f_2 \notin I(Y)$. We want to show that $f_1f_2 \in I(Y)$. Let $p \in Y$ be an arbitary point in $Y \subseteq \P^n$. Since $Y = C_1 \cup C_2$, then $p \in C_1$ or $p \in C_2$. If $p \in C_2$, then since $f_2 \in I(C_2) \setminus I(Y)$, then $f_1f_2(p) = 0$, i.e. it vanishes at $p$. Similarly, if $p \in C_1$ then the $f_1$ part of $f_1f_2$ vanishes at $p$, i.e. $f_1f_2(p) = f_1(p)f_2(p) = 0\cdot f_2(p) = 0$, and hence all of $f_1f_2(p) = 0$. Either case, we get that $f_1f_2 \in I(Y)$. Finally, since $I(Y)$ is a prime ideal, then this implies that either $f_1 \in I(Y)$ or $f_2 \in I(Y)$ which contradicts our earlier $f_1, f_2 \notin I(Y)$. Thus, $Y$ must be irreducible.

($\Rightarrow$) Suppose $Y \subseteq \P^n$ is irreducible. Let $f_1f_2 \in I(Y)$. We want to show that $f_1 \in I(Y)$ or $f_2 \in I(Y)$. Suppose for contradiction that $f_1, f_2 \notin I(Y)$, i.e. there exists some points $p_1, p_2 \in Y$ such that $f_1(p_1) \neq 0$ and $f_2(p_2) \neq 0$. Note that if $p_1 = p_2$, then we immediately reach a contradiction since $f_1f_2(p_1) = f_1(p_1)f_2(p_1) \neq 0$ contradicts $f_1f_2 \in I(Y)$. Therefore, we have $p_1 \neq p_2$. Note that since $f_1f_2 \in I(Y)$ then $f_1f_2$ vanishes at both $p_1$ and $p_2$, i.e. $f_1f_2(p_1) = 0$ and $f_1f_2(p_2) = 0$. Let's focus on $f_1f_2(p_1) = 0$. Since $f_1f_2(p_1) = f_1(p_1)f_2(p_1)$ and $f_1(p_1) \neq 0$, then $f_2(p_1) = 0$. Similarly, $f_1f_2(p_2) = 0$ implies $f_1(p_2) = 0$. Consider the closed sets $C_1 = Y \cap V(f_1)$ and $C_2 = Y \cap V(f_2)$. Note that $p_2 \in C_1$ and $p_1 \in C_2$. Next, let's consider $C_1 \cup C_2$, where:
$$ C_1 \cup C_2 = (Y \cap V(f_1)) \cup (Y \cap V(f_2)) = Y \cap (V(f_1) \cup V(f_2)). $$
Furthermore, note that $V(f_1f_2) = V(f_1) \cup V(f_2)$, hence:
$$ C_1 \cup C_2 = Y \cap V(f_1f_2), $$
but since $f_1f_2 \in I(Y)$, then this implies that $Y \subset V(f_1f_2)$, which means that $Y \cap V(f_1f_2) = Y$, thus
$$ C_1 \cup C_2 = Y. $$
Note that since $Y$ is irreducible, then this means that either $C_1 = Y$ or $C_2 = Y$. Without loss of generality, if $C_1 = Y$, then this means that $Y \subseteq V(f_1)$ which contradicts our assumption that $f_1 \notin I(Y)$. Therefore, $I(Y)$ \emph{must} be a prime ideal.

    \item First, note that $I(\P^n)$ is the ideal of homogeneous polynomials that vanish at every point in $\P^n$. Consider the \emph{homogeneous constant} polynomial $f(x_0, \ldots, x_n) = 0$. This is the only homogeneous polynomial in $k[x_0, \ldots, x_n]$ that vanishes at all points in $\P^n$, hence $I(\P^n) = (0)$. To show that $I(\P^n)$ is a prime ideal, note that
$$ k[x_0, \ldots, x_n] / (0) \cong k[x_0, \ldots, x_n], $$
and since $k[x_0, \ldots, x_n]$ itself is an integral domain, this implies that $I(\P^n) = (0)$ is a prime ideal. From \textbf{1.2.4b}, we know that $I(\P^n)$ being prime implies that $\P^n$ is irreducible. \qed


\end{enumerate}
\subsection*{2.10}
\addcontentsline{toc}{subsection}{2.10}

\textit{The Cone Over a Projective Variety} Let $Y \subseteq \P^n$ be a nonempty algebraic set, and let $\theta: \A^{n+1} \setminus \{(0, \ldots, 0)\} \to \P^n$ be the map which sends the point with affine coordinates $(a_0, \ldots, a_n)$ to the point with homogeneous coordinates $(a_0, \ldots, a_n)$. We define the \textit{affine cone} over $Y$ to be
$$C(Y) = \theta^{-1}(Y) \cup \{(0, \ldots, 0)\}.$$

\begin{enumerate}
    \item Show that $C(Y)$ is an algebraic set in $\A^{n+1}$, whose ideal is equal to $I(Y)$, considered as an ordinary ideal in $k[x_0, \ldots, x_n]$.
    \item $C(Y)$ is irreducible if and only if $Y$ is.
    \item $\dim C(Y) = \dim Y + 1$.
\end{enumerate}

\subsection*{2.10 (a) (b) {edealba}}

\begin{enumerate}
    \item We know that $Y \subseteq \P^n$ is a nonempty algebraic set, so there exists a homogeneous ideal $J$ such that $Z(J) = Y$. Note that since $Y = Z(J)$, then $C(Y) = C(Z(J))$. In \textbf{1.2.1} we showed that $C(Z(J)) = V(J) \subseteq \A^{n+1}$ for any ideal $J \subseteq k[x_0, \ldots, x_n]$, hence $C(Y) = V(J) \subseteq \A^{n+1}$. $C(Y) = V(J)$ is an algebraic set in $\A^{n+1}$ like we wanted to show.

Next, let's show that the ideal of $C(Y)$, i.e. $I(C(Y))$, is equal to $I(Y)$.

First, let's show that $I(Y) \subseteq I(C(Y))$. Let $f \in I(Y)$ such that $f$ is a \emph{homogeneous} polynomial of degree $d$, then this means that $f$ is a polynomial that vanishes at every point $p \in Y$. We want to show that $f \in I(C(Y))$. Let $q \in C(Y)$ be an arbitary point in the affine cone over $Y$. First case, suppose $q = (0, \ldots, 0)$. Since $I(Y)$ is the set of polynomials that vanish at every point in $Y \subseteq \P^n$, then these must be polynomials with \emph{zero} constant, hence $f(0, \ldots, 0) = 0$ (this is also clear since $f$ is homogeneous). Suppose $q = (a_0, \ldots, a_n)$ is an arbitary non-zero point in $C(Y)$ with some non-zero $a_i$. Note that, by definition of $I(Y)$, since $f$ vanishes at every \emph{projective} point in $Y$, then $f$ must vanish at \emph{all} of its representative points that exist in affine space. $q$ has a corresponding projective point $\overline{q} \in \P^n$ such that $f(\overline{q}) = 0$ which implies that $f(q) = 0$. If $\overline{q} = [q_0: \cdots : q_n] \in \P^n$, then $q$ is of the form:
$$ q = (\lambda q_0, \ldots, \lambda q_n), \text{ for } \lambda \in k, $$
and since $f$ is a homogeneous polynomial of degree $d$, then we can write:
$$ f(q) = f(\lambda q_0, \ldots, \lambda q_n) = \lambda^d f(q_0, \ldots, q_n) = \lambda^d \cdot 0 = 0, $$
so we see that any homogeneous polynomial $f \in I(Y)$ is in $I(C(Y))$. Furthermore, if we have some other \emph{non-homogeneous} polynomial $g \in I(Y)$, since $I(Y)$ is a homogeneous ideal, this means that it's generated by a set of homogeneous polynomials. Suppose $g = \sum_i h_i$, where $h_i$ is a homogeneous polynomial. Since each $h_i$ component vanishes for any $q \in C(Y)$ (since $h_i \in I(C(Y))$ for each homogenous $h_i$), then this means that $g$ overall also vanishes at $q$, hence $g \in I(C(Y))$. This shows that $f \in I(Y)$ then $f \in I(C(Y))$, thus $I(Y) \subseteq I(C(Y))$.

Second, le't show that $I(C(Y)) \subset I(Y)$. Let $f \in I(C(Y))$. Since $f$ vanishes at all points in $C(Y)$, and $(0, \ldots, 0) \in C(Y)$, then we know $f(0, \ldots, 0) = 0$, so like before, we know that $f$ is a polynomial with \emph{zero} constant term. $f$ is a graded ring over homogeneous polynomials, so we can write:
$$ f = \sum_i h_i,$$
where $h_i$ is a homogeneous polynomial of degree $i$. Note that if a point $q \in C(Y)$ then the entire line is in $C(Y)$, i.e. $\{ \lambda q : \lambda \in k \} \subseteq C(Y)$. Since $f \in I(C(Y))$, then we know that $f(\lambda q) = 0$ for all $\lambda \in k$. This means that:
$$ f(\lambda q) = \sum_i^n h_i(\lambda q) = \sum_i^n \lambda^i h_i(q) = 0, $$
and we can view this as a polynomial $g(\lambda) = f(\lambda q)$ where the $h_i(q)$ terms are seen as coefficients to $\lambda$. Note that:
$$ g(\lambda) =  \sum_i^n \lambda^i h_i(q) = 0, $$
for all $\lambda \in k$ implies that $g$ has infinitely many roots. However, note that by the Fundamental Theorem of Algebra, $g$ is a non-zero polynomial of degree $n$ and must have at most $n$ many roots, which is a contradiction. $g(\lambda) = 0$ for all $\lambda \in k$ if and only if $g$ is the \emph{zero polynomial}. This means that the coefficients of $g$ are all zero, i.e. $h_i(q) = 0$ for all $i$. With this, we've shown that all the homogeneous components of $f$ must vanish at any point $q \in C(Y)$. Since each $h_i$ is a homogeneous polynomial that vanishes at any point $q \in C(Y)$, then this means that for any projective point $[q] \in Y$, $h_i$ vanishes on any of its corresponding coordinates $q$, which implies that $h_i \in I(Y)$. Since $f = \sum_i h_i$, then $f$ also vanishes for any projective point $[q] \in Y$, i.e. $f \in I(Y)$. Finally, we see that $I(C(Y)) \subset I(Y)$ like we wanted to show.

Putting everything together, we see that, indeed, $I(C(Y)) = I(Y)$. \qed

    \item ($\Rightarrow$) Suppose $C(Y)$ is irreducible. From (a), we know that the ideal of $C(Y)$ is $I(Y)$, and since $C(Y)$ is irreducible, then $I(Y)$ must be a prime ideal. From \textbf{1.2.4b}, we know that $I(Y)$ being prime implies that $Y$ itself must be irreducible.

($\Leftarrow$) Suppose $Y$ is irreducible. Then this means that $I(Y)$ is a prime ideal. Furthermore, since $I(Y)$ is the ideal of $C(Y)$, i.e. $I(C(Y)) = I(Y)$, then this also implies that $C(Y)$ itself is irreducible. \qed

\end{enumerate}
\subsection*{2.10 (c) {azumaril.747}}

First we prove the case when $Y$ is irreducible. 

Let $\alpha_i:\A^n\to \A^{n+1}$ be the embedding defined by $\alpha_i((x_1,\ldots,x_n))=(x_1,\ldots,1,\ldots,x_n)$, where the coordinate component $1$ is on the $i$-th position. 

Let $U_i=\P^n-H_i$ ($i=0,1,\ldots,n$) be the open subsets of $\P^n$ defined in the text, with the homeomorphisms $\varphi_i:U_i\to\A^n$. By Corollary 2.3, we have $Y=\cup_i (Y\cap U_i)$. It follows that $\dim Y=\dim Y\cap U_i$ for some $i$, by Exercise 1.10 (b). Now consider the composition of maps $Y\cap U_i\xrightarrow{\varphi_i}\A^n \xhookrightarrow{\alpha_i}\A^{n+1}$, which sends $(x_0,x_1,\ldots,x_n)\in Y\cap U_i$ to $(\frac{x_o}{x_i},\ldots,1,\ldots,\frac{x_n}{x_i})\in \A^{n+1}$. The composition $\alpha_i\circ\varphi_i$ is an embedding with the image $C(Y)\cap Z(x_i-1)\subseteq \A^{n+1}$. So we have $\dim Y=\dim (C(Y)\cap Z(x_i-1))$. By Exercise 1.8, since $Z(x_i-1)$ is a hypersurface in $\A^{n+1}$, we have $\dim (C(Y)\cap Z(x_i-1))=\dim C(Y)-1$. So $\dim Y=\dim C(Y)-1$. 

When $Y$ is not irreducible, we decompose $Y$ into irreducible components $Y=\cup_i Y_i$. So, by definition, we have $C(Y)=\theta^{-1}(Y)\cup\{(0,\ldots,0)\}=\theta^{-1}(\cup_i Y_i)\cup\{(0,\ldots,0)\}=\cup_i \theta^{-1}(Y_i)\cup\{(0,\ldots,0)\}=\cup_i C(Y_i)$. This gives an irreducible cover of $C(Y)$, by Exercise 2.10 (b). We have $\dim C(Y)=\max \dim C(Y_i)=\max(\dim Y_i+1)=\max(\dim Y_i)+1=\dim Y +1$, so we finish the proof. 

Details of $\alpha_i\circ\varphi_i(Y\cap U_i)=C(Y)\cap Z(x_i-1)$: Let $\a:=I(Y)$ be the ideal of vanishing polynomials on $Y$. Suppose that $\a$ is generated by finitely many homogeneous polynomials $\{f_1,\cdots,f_r\}$. Then $Z^{\P}(\a)=Y$ and $Z^{\A}(\a)=C(Y)$, by Exercise 2.10 (a). We may assume without loss of generality that $i=0$. Then $C(Y)\cap Z(x_0-1)$ is the zero locus of the polynomials $\{f_1(x_0,x_1,\cdots,x_n),\cdots,f_r(x_0,x_1,\cdots,x_n), x_0-1\}$, or equivalently the zero locus of $\{f_1(1,x_1,\cdots,x_n),\cdots,f_r(1,x_1,\cdots,x_n), x_0-1\}$. Notice that $f_i(1,x_1,\cdots,x_n)$ is just $\alpha(f_i)$, and $\varphi_0(Y\cap U_0)$ is exactly $Z(\alpha(f_1),\cdots,\alpha(f_r))\subseteq \A^n$, where $\alpha$ is the notation in the proof of Proposition 2.2. So the claim follows. 

Remark: My intuition is that if we truncate the cone $C(Y)$ by the hyperplane $x_i=1$, the intersection should be isomorphic to $Y\cap U_i$. Cutting by a hyperplane reduces the dimension of $C(Y)$ by $1$.  
    \subsection*{2.12}
    \addcontentsline{toc}{subsection}{2.12}
	{\em The $d$-{\rm Uple} Embedding}.  For given $n, d >0$, let $M_0, M_1, \ldots, M_N$ be all the monomials of degree $d$ in the $n+1$ variables $x_0, \ldots, x_n$, where $N = {n+d \choose n} - 1$.  We define a mapping $\rho_d \colon \mathbb{P}^n \to \mathbb{P}^N$ by sending the point $P = (a_0, \ldots, a_n)$ to the point $\rho_d(P) = (M_0(a), \ldots, M_N(a))$ obtained by substituting the $a_i$ in the monomials $M_j$.  This is called the {\em $d$-{\rm uple} embedding} of $\mathbb{P}^n$ in $\mathbb{P}^N$.  For example, if $n=1,d=2$, then $N=2$, and the image $Y$ of the 2-uple embedding of $\mathbb{P}^1$ in $\mathbb{P}^2$ is a conic.

\begin{enumerate}
	\item  Let $\theta \colon k[y_0, \ldots, y_N] \to k[x_0, \ldots, x_n]$ be the homomorphism defined by sending $y_i$ to $M_i$, and let $\mathfrak{a}$ be the kernel of $\theta$.  Then $\mathfrak{a}$ is a homogeneous prime ideal, and so $Z(\mathfrak{a})$ is a projective variety in $\mathbb{P}^N$.
	\item Show that the image of $\rho_d$ is exactly $Z(\mathfrak{a})$.  (One inclusion is easy.  The other will require some calculation.)
	\item Now show that $\rho_d$ is a homeomorphism of $\mathbb{P}^n$ onto the projective variety $Z(\mathfrak{a})$.
	\item Show  that the twisted cubic curve in $\mathbb{P}^3$ (Ex. 2.9) is equal to the 3-uple embedding of $\mathbb{P}^1$ in $\mathbb{P}^3$, for suitable choice of coordinates.
\end{enumerate}
    \subsection*{2.12 (b)   \href{https://github.com/mikeleew}{mlwells}
}


Let $Q = \rho_d(P) = (M_0(a), \ldots, M_N(a))$.  Let $F \in \mathfrak{a} = \mbox{\rm ker }\theta$.  Then $F(M_0, \ldots, M_N) = 0$ as a polynomial, which implies $F(M_0(a), \ldots, M_N(a)) = 0$.  Since $F$ was arbitrary, $Q \in Z(\mathfrak{a})$. 

Now let $Q = (y_0, \ldots, y_N) \in Z(\mathfrak{a})$.  Suppose without loss of generality that $y_0 \neq 0$.  It follows that $Q = (y_0^d, y_0^{d-1} y_1, \ldots, y_0^{d-1} y_N)$.  Suppose that $M_0 = x_0^d, M_1 = x_0^{d-1} x_1, \ldots, M_n = x_0^{d-1} x_n$.
\setcounter{section}{1}
\setcounter{subsection}{2}
\begin{proposition}
	For all $j =0,\ldots, N$, we have that 
	\begin{equation}
		M_0^{d-1} M_j = M_0^{i_0} M_1^{i_1} \ldots M_n^{i_n} 
	\end{equation}
	for some $i_j \geq 0$ with $\sum_j i_j = d$.
\end{proposition}

\begin{proof}
Suppose $M_j = x_0^{i_0} \ldots x_n^{i_n}$ where $\sum_j i_j = d$.  We have
\begin{align}
M_0^{d-1} M_j &= x_0^{(d-1)d + i_0} x_1^{i_1} \ldots x_n^{i_n}\\
&= x_0^{(d-1)i_0 + i_0 + (d-1)\sum_{j \geq 1} i_j} x_1^{i_1} \ldots x_n^{i_n}\\
&= x_0^{d i_0} (x_0^{d-1} x_1)^{i_1} \ldots (x_0^{d-1} x_n)^{i_n}\\
&= M_0^{i_0} M_1^{i_1} \ldots M_n^{i_n}
\end{align}
\end{proof}

Let $P = (y_0, \ldots, y_n)$.  Then, by the proposition, we have that $Q = (M_0(P), M_1(P), \ldots, M_N(P)) = \rho_d(P)$, as was to be shown.

\section*{1.3 Morphisms}
\addcontentsline{toc}{section}{1.3 Morphisms}
\subsection*{3.20}
\addcontentsline{toc}{subsection}{3.20}

Let $Y$ be a variety of dimension $\geq 2$, and let $P \in Y$ be a normal point.  Let $f$ be a
regular function on $Y - P$.
\begin{enumerate}
    \item Show that $f$ extends to a regular function on $Y$.
    \item Show this would be false for $\mbox{rm dim }Y = 1$.

    See (III, Ex. 3.5) for generalization.
\end{enumerate}


    \subsection*{3.20 (a)   \href{https://github.com/mikeleew}{mlwells}
}


If we show that $f$ restricted to $V - P$ extends to a regular function on $V \subseteq Y$ for any affine neighborhood $V$ of $P$,
then by gluing we will have shown that $f$ extends to a regular function on $Y$.  So, assume that $Y$ is affine.

The function $f = c/d \in K(Y)$ for regular functions $c, d \in A(Y)$.  If we can show that the ideal quotient
$((c) : (d)) = \{h \in A(Y) \colon hc \in (d)\}$ is equal to $A(Y)$, then it follows that $c/d \in A(Y)$.  By assumption,
$f \in \mathcal{O}_Q$ for all $ Q \neq P \in Y$.  Let $\mathfrak{m}_Q$ denote the maximal ideal of $A(Y)$ corresponding to $Q$.
Then the localized ideal $((c) : (d))_{\mathfrak{m}_Q}$ is equal to $A(Y)_{\mathfrak{m}_Q} = \mathcal{O}_Q$ for all $Q \neq P$ since $c/d \in \mathcal{O}_Q$.  It remains
to show that $((c) : (d))_{\mathfrak{m}_P} = A(Y)_{\mathfrak{m}_P} = \mathcal{O}_P$.  If $\mathfrak{p} A(Y)_{\mathfrak{m}_P}$ is a prime ideal of
$\mathcal{O}_P$ not equal to the maximal ideal $\mathfrak{m}_P A(Y)_{\mathfrak{m}_P}$, then $((c) : (d))_{\mathfrak{m}_P} \not\subseteq \mathfrak{p}A(Y)_{\mathfrak{m}_P}$
since $c/d$ is in the local ring corresponding to the subvariety defined by $\mathfrak{p}$, and hence
$((c) : (d))_{\mathfrak{p}} = A(Y)_{\mathfrak{p}}$.
.  Assume by way of contradiction
that $((c) : (d))_{\mathfrak{m}_P} \subseteq \mathfrak{m}_P A(Y)_{\mathfrak{m}_P}$. Then, $\sqrt{((c) : (d))_{\mathfrak{m}_P}} = \mathfrak{m}_P$.

Let $a_1, \ldots, a_s$ be a regular sequence in $\mathfrak{m}_P A(Y)_{\mathfrak{m}_P}$ with $s > 1$.  By Theorem 8.22A of
Chapter II of Hartshorne part (2), such a sequence exists.  We have shown that $a_1^{r_1}, a_2^{r_2} \in ((c): (d))_{\mathfrak{m}_P}$
for some $r_1, r_2 > 0$ with $a_1^{r_1-1}, a_2^{r_2-1} \notin ((c) : (d))_{\mathfrak{m}_P}$  .  Thus,
\begin{equation}
    a_1^{r_1} c = e_1 d, \;\; e_1 \notin (a_1)
\end{equation}
and
\begin{equation}
    a_2^{r_2} c = e_2 d, \;\; e_2 \notin (a_2)
\end{equation}
Thus,
\begin{equation}
    a_1^{r_1} a_2^{r_2} c = e_1 a_2^{r_2} d = e_2 a_1^{r_1} d
\end{equation}
which implies
\begin{equation}
    e_1 a_2^{r_2} = e_2 a_1^{r_1}
\end{equation}
Since $e_1 \notin (a_1)$ and $a_2 \notin (a_1)$, $a_2$  is a zero divisor in the ring $A(Y)_{\mathfrak{m}_P}/(a_1)$.  This
contradicts the fact that $a_1, a_2$ is a regular sequence in $\mathfrak{m}_P A(Y)_{\mathfrak{m}_P}$.

Thus, we conclude that $((c) : (d))_{\mathfrak{m}_P} = A(Y)_{\mathfrak{m}_P}$, and hence $f \in \mathcal{O}_P$.  This
shows that $((c): (d)) = A(Y)$ which implies $f \in A(Y)$, and hence $f$ extends to a regular function on $Y$.
\section*{2.1 Sheaves}
\addcontentsline{toc}{section}{2.1 Sheaves}

\phantomsection
\addcontentsline{toc}{subsection}{1.2}
\subsection*{1.2}
\label{subsec:II.1.2}


\begin{enumerate}
    \item For any morphism of sheaves $\phi: \mathscr{F}\to\mathscr{G}$, show that for each point $P$, $(\ker \phi )_{P}=\ker(\phi_{P})$ and $(\operatorname{im}\phi)_{P} = \operatorname{im}(\phi_{P})$.
    \item Show that $\phi$ is injective (surjective) iff induced map on stalks $\phi_{P}$ is injective (surjective) for all $P$.
    \item Show that the sequence of sheaves $\begin{tikzcd}[cramped, sep=small] \cdots \arrow[r] & \mathscr{F}^{i-1} \arrow[r] & \mathscr{F}^{i} \arrow[r] & \mathscr{F}^{i+1} \arrow[r] & \cdots \end{tikzcd}$

\end{enumerate}
    \subsection*{1.2   \href{https://github.com/yakimk}{yakimk}
}

\begin{enumerate}
    \item Since stalks are defined as a filtered colimit and kernels are a particular type of limit (pullback) and \href{https://stacks.math.columbia.edu/tag/002W}{filtered colimits commute with finite limits} isomorphism $(\ker \phi_{P}) = \ker(\phi_{P})$ is automatic.

To show a similar thing for the image we first note that $(\operatorname{im}\phi)_{P} \simeq ({(\operatorname{im}_{pre}\phi)^{+})_{P}}\simeq (\operatorname{im}_{pre} \phi)_{P}$, where the first isomorphism is by definition and the second one using the fact that sheafification has the same stalks as original presheaf.  Now we conclude noticing that $\operatorname{im}_{pre}\phi \simeq \ker(\operatorname{coker}\phi)$, i.e. it is a finite limit of finite colimits and since again stalks are filtered colimits they commute with finite limits and colimits (in for instance abelian category) we conclude that $(\operatorname{im} \phi)_{P} \simeq \operatorname{im}(\phi_{P})$.

\item We show that a map of sheaves $\phi: \mathscr{F} \to \mathscr{G}$ is injective iff it induced maps on stalks are injective (proof for surjectivity is essentially identical).

Consider the following commutative diagram
\[
\begin{tikzcd}
\mathscr{F}(U)     \arrow[r,"\phi_{U}"] \arrow[d,"\iota _{1}"'] &  \mathscr{G}(U) \arrow[d,"\iota_{2}"]\\ 
\prod_{P \in U} \mathscr{F}_{p}     \arrow[r,"\prod \tilde{\phi }_{P}"] &  \prod_{P \in U}\mathscr{G}_{p}
\end{tikzcd}
\]

Note that since section of a sheaf on an open set is determined by its stalks, $\iota_{1}$ and $\iota_{2}$ are both injective.

Suppose $\tilde{\phi}_{P}$ is injective for all $P$. Take two sections $f, g \in \mathscr{F}(U)$, since $\prod \tilde{\phi}_{P}$ and $\iota_{1}$ are injective (former by the hypothesis that $\tilde{\phi}_{P}$ are injective), their composition is injective. If $\phi_{U}$ were not injective, we could find two different sections that go to the same class in $\prod \mathscr{G}_{P}$, which would contradict injectivity of $\prod \tilde{\phi}_{P} \circ \iota_{1}$.

Similarly if $\phi_{U}$ is injective then if some of $\tilde{\phi}_{P}$  were not injective, its composition with $\iota_{1}$ would not be injective contradicting commutativity of the diagram above. 

\item We have to show that 
$\begin{tikzcd}[cramped, sep=small] \cdots \arrow[r] & \mathscr{F}^{i-1} \arrow[r] & \mathscr{F}^{i} \arrow[r] & \mathscr{F}^{i+1} \arrow[r] & \cdots \end{tikzcd}$
is exact iff induced sequence of stalks are exact for all $P$.

Assuming that sequence of sheaves is exact we show induced sequences are exact at all stalks. By the previuos section $\operatorname{im} \phi_{P}^i \simeq (\operatorname{im} \phi^i)_{P}\simeq(\ker \phi^{i+1})_{P}\simeq \ker \phi_{P}^{i + 1}$.

Now assume that all sequences of stalks are exact.  By the same reasoning as before we see that  $\operatorname{im} \phi^{i}\simeq \ker \phi^{i+1}$ (since they are isomorphic on each stalk and hence are equal).
\end{enumerate}


\end{document}