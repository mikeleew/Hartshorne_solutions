\documentclass{article}
\usepackage[utf8]{inputenc}
\usepackage{amsmath,enumitem,amsfonts}
\usepackage{mathtools}
\usepackage{subcaption}
\usepackage{caption}
\usepackage{comment}
\usepackage{float}
\usepackage{biblatex}
\usepackage{amsthm}
\usepackage{amsfonts}
\usepackage{amssymb}
\usepackage[, total={6.5in, 8in}]{geometry}


\newtheorem{theorem}{Theorem}[section]
\newtheorem{lemma}{Lemma}[theorem]
\newtheorem{corollary}{Corollary}[theorem]
\newtheorem{proposition}{Proposition}[theorem]
\setlength{\parindent}{0cm}
\setcounter{section}{0}

\title{Hartshorne Solutions}
\author{mlwells} 
\date{2025}

\begin{document}
\maketitle
Feel free to add your Discord name to the author list.  Let's add our usernames to our solutions, too, so that people get proper credit.
\section*{Chapter 1}
\subsection*{1. Affine Varieties}
Solutions:
\begin{itemize}
    \item [] 1.1
    \begin{itemize}
        \item [] (a)
        \item [] (b)
        \item [] *(c)
    \end{itemize}
    \item [] 1.2
    \item [] 1.3
    \item [] 1.4
    \item [] 1.5
    \item [] 1.6
    \item [] 1.7
    \begin{itemize}
        \item [] (a)
        \item [] (b)
        \item [] (c)
        \item [] (d)
    \end{itemize}
    \item [] 1.8  Let $Y$ be an affine variety of dimension $r$ in $\mathbb{A}^n$.  Let $H$ be a hypersurface in $\mathbb{A}^n$, and assume that $Y \not\subseteq H$.  Then every irreducible component of $Y \cap H$ has dimension $r-1$.

    mlwells:
    
    Suppose that $H = Z(f)$ with $f$ irreducible in $k[x_1,\ldots, x_n]$.  Then the projection of $f$ in $A(Y) := k[x_1,\ldots,x_n]/I(Y)$ is not equal to $\overline{0}$ since by assumption $(f) \not\subset I(Y)$.  Since $A(Y)$ is a domain (due to the irreducibility of $Y$), the element $\overline{f}$ is not a zero divisor.  Assuming that $Y \cap H \neq \varnothing$, we have that $\overline{f}$ is not a unit in $A(Y)$.  To see this, let $P \in Y \cap H$.  Then $I(Y), (f) \subset \mathfrak{m}_P$, the maximal ideal of $k[x_1,\ldots,x_n]$ corresponding to $P$.  This implies $(\overline{f}) \subset \mathfrak{m}_P / I(Y)$, the latter being a maximal ideal in $A(Y)$.  Thus, $\overline{f}$ is not a unit.

    We apply Theorem 1.11A to get that every minimal prime ideal $\mathfrak{p}$ in $A(Y)$ containing $\overline{f}$ has height 1.  The irreducible components of $Y \cap H$ and the minimal prime ideals containing $(\overline{f})$ correspond, and since each of these prime ideals has height 1, the corresponding varieties have dimension $r-1$ by Theorem 1.8A.
    \item [] 1.9
    \item [] 1.10
    \begin{itemize}
        \item [] (a)
        \item [] (b)
        \item [] (c)
        \item [] (d)
        \item [] (e)
    \end{itemize}
    \item [] *1.11
    \item [] 1.12
\end{itemize}


\end{document}